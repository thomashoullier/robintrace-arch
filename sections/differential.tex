\section{Differential raytracing}
We aim to implement a way to provide the ray data along with the derivatives of
these scalars with respect to the design variables.

The interplay with ray-aiming is hard to foresee. We fear the ray-aiming
disrupts the gradient from the image MF to the design parameters.

Keep in mind:

\begin{itemize}
\item The ray-aiming itself is more easily solved with gradient data.
\item We may maintain an explicit representation of the ray-aiming relationship
between entrance and pupil (essentially a paraxial approximation of raytracing)
which is updated throughout the optimization and included in the gradient
computations for the MF.
\end{itemize}

We can implement the raytracing of poaky using an automatic differentiation
engine such as: autodiff (C++, MIT), clad (C++ for Clang, LGPL). We'll try
with autodiff, which seems also compatible with Eigen.

There is difficulty in managing the differential results when ray-aiming is
in the mix. In any case, we can implement the Robintrace parts which
compute both the ray-aiming ray data along with derivatives with respect
to ray entrance heights, and the full ray data with derivatives with respect
to design variables. We can worry about soring this out with optimizers later
(maybe read about joint optimization, constrained optimization, bi-level
optimization etc.).
