\section{Architecture}
The software architecture follows an object-oriented, bottom-up approach.
From the lowest level of abstraction to highest, we may describe the
major objects as follows:

\begin{itemize}
\item \lstinline{ray}: Rays are the most fundamental data object in the
program.  Each ray is represented by $(x, y, z)$ local coordinates and $(l, m,
n)$ local direction cosines.
\item \lstinline{rop} (ray operations): These are low-level operations
modifying \lstinline{rays}.  They are used by \lstinline{low-level surfaces}.
They may be common to multiple surface types. A typical example is the
Snell law.
\item \lstinline{shape}: These are geometric surface shapes. They are used
mainly for specifying intersection methods with rays, and respond to queries
about normal vector and altitude at given $(x, y)$ coordinates.
\item \lstinline{lpart} (low-level part): These are low-level objects which
are ordered in succession. These are groupings of \lstinline{rop}
which correspond to common optical design surface elements. There are two
fundamental subtypes:
\begin{itemize}
\item \lstinline{surf} (surface): This group of operations takes a ray starting
from the local surface plane and outputs rays also in the local coordinate
system at the computed positions. May inherit a \lstinline{shape}.
\item \lstinline{tfr} (transfer): These operations take rays from a starting
surface coordinate system and propagate them in straight lines to another
plane. The output rays are expressed in a new local surface plane.
\end{itemize}
\item \lstinline{bun} (ray bundles): These are arrays of \lstinline{ray}. They
include some higher level utilities such as tracking and reporting the state of
rays.
\item \lstinline{lsys} (low-level system): This is a succession of
\lstinline{lpart}.
\item \lstinline{ltrac} (low-level tracer): This object holds both a
\lstinline{lsys} and a \lstinline{bun}. It is responsible for applying
operations and reporting the state of the rays to higher parts of the program.
\end{itemize}

This architecture is summarized informally using a diagram
(\cref{fig:arch-overview}).

\begin{figure}
\includesvg[width=.9\textwidth]{images/arch/overview/overview.svg}
\caption{\label{fig:arch-overview} Architecture overview.}
\end{figure}
