\section{Ray-aiming}
We document some possible concepts for solving ray-aiming.

\begin{itemize}
\item Using a paraxial approximation as a starting point, or a development
      in increasing orders of accuracy as shown in \cite{Zheng2010}.
\item Tracing only cardinal rays, or rays along the edge of the aperture
      stop and filling it with an a priori model, as in \cite{Houllier-thesis}.
\item Computing an approximate mapping between entrance and aperture stop,
      using paraxial optics or sampling.
\item Direct inversion of the surfaces' raytracing (may be impossible).
\item Implicit ray-aiming, include it as an objective in the MF. In the
      cases we do not need the raytracing to be very accurate in the MF
      evaluations. Would work well for local search algorithms. The
      entrance ray heights would be free variables.
\item Train a NN to do the ray-aiming. The training could be done on
      a very robust but slow method.
\item Start the ray-aiming with surfaces which cannot be missed by rays (such as
      paraboloids) and gradually change the surfaces back to their original shape.
      This is probably infeasible.
\end{itemize}
