\section{Scope}
The development of RobinTrace aims at reaching a minimal set of features which
make it a usable tool for optical design of simple systems.

The aims are as follows:
\begin{itemize}
\item \textbf{Minimal}: RobinTrace deals with sequential raytracing. Matters of
assessing optical performance from ray states is higher level and out of the
scope.
\item \textbf{For programmers}: We build an API for raytracing bundles of
rays through optical parts sequences and report the ray states along the way.
The user is expected to be familiar with both C++ and optical design.
\item \textbf{Efficient}: The runtime execution should be fast. An
execution speed in the same ballpark as that of commercial tools seems
a reasonable goal.
\end{itemize}

The following subjects are part of RobinTrace:
\begin{itemize}
\item Raytracing equations: \emph{eg} intersecting spheres and quadrics of
revolution, refraction and reflection.
\item Abstractions which suit optical designers, who are interested in
getting ray data for a particular system and particular field and pupil
sampling.
\item Exposing the ray data and gradients with respect to design
parameters for use by the user in optimization frameworks.
\item Ray-aiming is included in the features as an exposed optimization
problem. The user is free to treat this as a constraint or bi-level
optimization etc.
\end{itemize}

The following subjects are put on hold until later in the development:
\begin{itemize}
\item Other physical properties of rays: phase, path length, polarization,
amplitude and intensity.
\item Other optical parts: freeform shapes, gratings.
\item Jacobian and Hessian of ray data with respect to optical system
parameters. See differential raytracing. This remains easy while the
number of surface types is limited.
\item Short-hands and utilities for defining sequences of optical parts,
defining fields etc. These will be implemented if the need becomes apparent.
\item (still debated) Utilities for defining aperture stop samplings.
\item Visualisations, such as layout diagrams (2D or 3D) provide great
graphical feedback on results and are a good way to discover bugs. They also
likely require specific raytracing which the RobinTrace user cannot be
expected to implement.
\item Mapping operations onto multiple threads or GPU architectures. This
is a hard topic since sequential raytracing is cheap and the threshold
at which parallelism becomes worthwhile might be as high in the abstraction
ladder as the search algorithms running on the optical design problems.
\item A Python API wrapper. This is meant to make RobinTrace accessible
to higher abstraction utilities and existing Python libraries. This could be
implemented with swig.
\end{itemize}

The following optical design related items are left out:
\begin{itemize}
\item Building objective functions and constraints: This belongs in a higher
layer which uses the reported states of rays to build performance criteria.
Since the objectives are numerous and can be specific, we leave their
computation up to the user.
\item In the same fashion, assessments of optical performance
(breakdown in aberrations etc.) which can be built on top of raytracing
are not planned for implementation in RobinTrace and belong in other
programs.
\item Optimization: This is an even higher layer than objective functions.
We want the user to be able to interface with the optimization algorithm
of their choice.
\item GUI for input of optical systems and control of the program. This
is a huge work which we have no interest in.
\end{itemize}

Many concepts relevant to raytracing have been explored in depth in a previous
work \cite{Houllier-thesis}.

